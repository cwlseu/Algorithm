% !Mode:: "TeX:UTF-8"
\documentclass{article}
\usepackage{lastpage} % Required to determine the last page for the footer
\usepackage{extramarks} % Required for headers and footers
\usepackage{graphicx} % Required to insert images
\usepackage{listings} % Required for insertion of code
\usepackage[usenames,dvipsnames]{color} % Required for custom colors
\usepackage{amsmath,amsthm}
\usepackage{courier} % Required for the courier font
\usepackage{enumerate}
\usepackage{algorithm}
\usepackage{algorithmic}
\usepackage{algpseudocode}
\usepackage{supertabular}
% Margins
\linespread{1.3} % Line spacing
% Set up the header and footer

%----------------------------------------------------------------------------------
%   CODE INCLUSION CONFIGURATION
%----------------------------------------------------------------------------------------
\definecolor{MyDarkGreen}{rgb}{0.0,0.4,0.0} % This is the color used for comments
\lstloadlanguages{Python} % Load Perl syntax for listings, for a list of other languages supported see: ftp://ftp.tex.ac.uk/tex-archive/macros/latex/contrib/listings/listings.pdf
\lstset{language=Python, % Use Perl in this example
        %frame=single, % Single frame around code
        basicstyle=\small\ttfamily, % Use small true type font
        keywordstyle=[1]\color{Blue}\bf, % Perl functions bold and blue
        keywordstyle=[2]\color{Purple}, % Perl function arguments purple
        keywordstyle=[3]\color{Blue}\underbar, % Custom functions underlined and blue
        identifierstyle=, % Nothing special about identifiers
        commentstyle=\usefont{T1}{pcr}{m}{sl}\color{MyDarkGreen}\small, % Comments small dark green courier font
        stringstyle=\color{Purple}, % Strings are purple
        showstringspaces=false, % Don't put marks in string spaces
        tabsize=3, % 5 spaces per tab
        %
        % Put standard Perl functions not included in the default language here
        morekeywords={rand},
        %
        % Put Perl function parameters here
        morekeywords=[2]{on, off, interp},
        %
        % Put user defined functions here
        morekeywords=[3]{test},
        %
        morecomment=[l][\color{Blue}]{...}, % Line continuation (...) like blue comment
        numbers=left, % Line numbers on left
        firstnumber=1, % Line numbers start with line 1
        numberstyle=\tiny\color{Blue}, % Line numbers are blue and small
        stepnumber=5 % Line numbers go in steps of 5
}

% Creates a new command to include a python script, the first parameter is the filename of the script (without .py), the second parameter is the caption
\newcommand{\pythonscript}[2]{
\begin{itemize}
\item[]\lstinputlisting[caption=#2,label=#1]{#1.py}
\end{itemize}
}

%----------------------------------------------------------------------------------------
%   NAME AND CLASS SECTION
%----------------------------------------------------------------------------------------

\newcommand{\hmwkTitle}{Assignment\ \#3} % Assignment title
\newcommand{\hmwkDueDate}{Thursday,\ Nov \ 5,\ 2015} % Due date
\newcommand{\hmwkClass}{CS091M4041H\ 101} % Course/class
\newcommand{\hmwkClassTime}{9:20am} % Class/lecture time
\newcommand{\hmwkClassInstructor}{Dongbo Bu} % Teacher/lecturer
\newcommand{\hmwkAuthorName}{cwlseu} % Your name

\numberwithin{equation}{section}
%----------------------------------------------------------------------------------------
%   TITLE PAGE
%----------------------------------------------------------------------------------------

\title{
\vspace{2in}
\textmd{\textbf{\hmwkClass:\ \hmwkTitle}}\\
\normalsize\vspace{0.1in}\small{Due\ on\ \hmwkDueDate}\\
\vspace{0.1in}\large{\textit{\hmwkClassInstructor\ \hmwkClassTime}}
\vspace{3in}
}

\author{\textbf{\hmwkAuthorName}\\\textbf{\hmwkAuthorStuID}}
\date{} % Insert date here if you want it to appear below your name

\begin{document}

\maketitle
\newpage


%
% Problem 1
%
\section{Greedy Algorithm-1}
Given a list of n natural numbers $d_1$, $d_2$, . . . , $d_n$, show how to decide in polynomial time whether there exists an undirected graph $G = (V, E)$ whose node degrees are precisely the numbers $d_1$, . . . ,$d_n$. $G$ should not contain multiple edges between the same pair of nodes, or \"loop\" edges with both endpoints equal to the same node.
    \subsection{Algorithm}
    If all the $d_i$ are 0, then we know that there is such a graph: the graph has $n$ vertices and no edges. So assume this is not the case. Sort the $d_i$ in decreasingorder: $d_1 \geq d_2 \geq . . . \geq d_n$. Now argue that there is a graph with degree sequence ($d_1$, . . . , $d_n$) if and only if there is a graph (on $n - 1$ vertices) with degree sequence ($d_2 - 1$, $d_3 - 1$, . . . ,$ d_k - 1$, $d_k+1$, . . . , $d_n$), where $k = d_1$. In other words, we are saying that if a graph with sequence ($d_1$, . . . , $d_n$) exists, then we can assume that the highest degree vertex (of degree $d_1$) has edges to the next $d_1$ highest degree vertices. 
    \alglanguage{pseudocode}
    \begin{algorithm}
    \caption{Undirected-Graph-With-Sequence-degrees}
    \label{alg1}
    \begin{algorithmic}[1]
    \STATE initialize a graph G with n vertices and no edges

    \STATE Sort the degrees sequences $L_0 \gets {d_1, d_2, ...,d_n}$ where $d_1 \geq d_2 \geq...\geq d_n$
    \STATE $L_1 \gets \{$d_2 - 1, d_3 - 1,...,d_k - 1, d_k+1,..., d_n$\} where k = d_1$
    \STATE Link the graph G's vertices $v_1$ with {$v_2, v_3,...,v_{d_k}$}
    \STATE $L_0 \gets L_1$ 

    \STATE \emph{REPEAT} Previous 4 steps \emph{UNTIL} $L_0 \eq NULL$  
    \end{algorithmic}
    \end{algorithm}


    \subsection{Correctless}
    If all the $d_i$ are 0, then we know that there is such a graph: the graph has $n$ vertices and no edges. So assume this is not the case. Sort the $d_i$ in decreasingorder: $d_1 \geq d_2 \geq...\geq d_n$. Now argue that there is a graph with degree sequence ($d_1$,... , $d_n$) if and only if there is a graph (on $n - 1$ vertices) with degree sequence ($d_2 - 1$, $d_3 - 1$,...,$ d_k - 1$, $d_k+1$,..., $d_n$), where $k = d_1$. In other words, we are saying that if a graph with sequence ($d_1$,..., $d_n$) exists, then we can assume that the highest degree vertex (of degree $d_1$) has edges to the next $d_1$ highest degree vertices. 
    Let us see why. One direction of the proof is easy: if there is a graph G with degree sequence ($d_2 - 1$, $d_3 - 1$,...,$ d_k - 1$, $d_k+1$,..., $d_n$), then there is a graph with degree sequence ($d_1$,..., $d_n$): add a new vertex to G which has edges to the vertices with degrees $d_2 - 1$, $d_3 - 1$,..., $d_k - 1$. Let us now prove the reverse (and the more non-trivial direction of the proof).
    Suppose there is a graph G with degree sequence ($d_1$,..., $d_n$).
    Let $v_i$ be the vertex with degree $d_i$. If $v_1$ has edges to $v_2$,..., $v_k$ in G, then we are done
    just remove $v_1$ and you have the graph with the desired degree sequence. So assume there is an index i, $2 \geq i \geq k$ such that ($v_1$, $v_i$) is not an edge. Since degree of $v_1$ is k(=$d_1$), there must be an index $j > k$ such that ($v_1$, $v_j$ ) is an edge. Since degree of $v_i$
    is at least that of $v_j$, there must be a vertex $v_k$ such that ($v_i$, $v_k$) is an edge, but ($v_j$, $v_k$) is not an edge. Now, in G, we remove the edges ($v_1, v_j$ ) and ($v_i$, $v_k$), and add the edges
    ($v_1$, $v_i$) and ($v_j$, $v_k$). Note that this does not change the degree of any vertex, but now,
    we have increased the number of edges from $v_1$ to the vertices in the set {$v_2$,..., $v_k$}.
    Repeat this process till $v_1$ has edges to {$v_2,...,v_k$}.
    \subsection{Complexity}
    The algorithm the most time used for sort the nuture number sequence in each loop.The sort time is the 
    $\sum{klogk}$,we can calculate that the time complexity is $O(n^2logn)$
%
% Problem 2
%
\section{Greedy Algorithm-4}
Suppose you are given two sets A and B, each containing n positive integers. 
You can choose to reorder each set however you like. After reordering, let $a_i$ be te $i$th element of set A, and let $b_i$ be the $i$th element of set B. 
You then receive a payoff of $\textstyle\Pi_{i=1}^n a_i^{b_i}$.Given an polynomial-time algorithm that will maximize your payoff.
    \subsection{Algorithm}
    Sort A and B both into decreasing order.Then calculate the payoff. 
    \alglanguage{pseudocode}
    \begin{algorithm}
    \caption{Maximize Two Set's Payoff}
    \label{alg2}
    \begin{algorithmic}[1]
    \STATE $result \gets 1$
    \STATE $A_1 = \emph{QuickSort(A)}   B_1 = \emph{QuickSort(B)}$
    
    \FOR{$i \gets 1, i < n$}
        \STATE $result \gets result * a_i^{b_i}$ 
        \Comment{$a_i = A_1[i] AND b_i = B_1[i]$}
    \ENDFOR
    \RETURN $result$
    \end{algorithmic}
    \end{algorithm}
    \subsection{Correctless}
    There are two set $A = (1, 2)$ and $B = (1, 2)$, we can see that $$4 = a_1^{b_1}*a_2^{b_2} > a_2^{b_1}*a_1^{b_2} = 2$$So,we select $(a_i,a_j)$ from A and $(b_i,b_j)$ from B that $a_i > a_j$ and $b_i > b_j$, it's obvious that $a_i^{b_i} * a_j^{b_j} \geq a_i^{b_j} * a_j^{b_i}$. So the maximize payoff could be calculate by this way that sort the set A and B, then calculate payoff one by one in sequence. 
    \subsection{Complexity}
    The algorithm time complexity is the $$T(n) = T_{sort} + T_{for} = O(nlogn) + O(n)$$  So the algorithm time complexity is $O(nlogn)$
%
% Problem 3
%
\section{Programming-5}
Write a program in your favorate language to compress a file using Huffman code and then decompress it. Code information may be contained in the the compressed file if you can. Use your program to compress the two files(\emph{gragh.txt} and \emph{Aesop\_Fables.txt}) and compare the results(Huffman code and compression ratio).
\subsection{Compression ratio}
compression file size is 105KB and 910KB where the original file \emph{Aesop\_Fables.txt} size is 186KB and \emph{graph.txt} size is 2046KB. The compression ratio is 56.5\% and 44.5\%.
\subsection{Huffman Code}
The huffman code as following table show. From the result, it's easy to see that the different frequence of letter lead to different huffman code length. Because the more frequent letter will use a short length huffman code. Above all, the \emph{graph.txt} use short 0,1 sequence to code number and \emph{Aesop\_Fables.txt} use long 0,1 sequence to code number.
            \begin{center}
            \tablefirsthead{%
            \hline
             ASCII Code & \emph{Aesop\_Fables.txt} & \emph{gragh.txt}  \\ 
            \hline}
            \tablehead{%
            \hline
             ASCII Code & \emph{Aesop\_Fables.txt} & \emph{gragh.txt}  \\ 
            \hline}
            \tabletail{%
            \hline}
            \tablelasttail{\hline}
            \bottomcaption{Huffman Code ASCII Letter}
            \begin{supertabular}{ccc}
                    0  &  101101010101101 &  01001  \\
                    1  &  0000101110110 &   00      \\
                    2  &  0000101110100 &  1100     \\
                    3  &  0000101110111 &  1010     \\
                    4  &  1011010101010 &   1001    \\
                    5  &  0000101110101 &   0110    \\
                    6  &  11001110110010 &  0111    \\
                    7  &  10110101010111 &   0101   \\
                    8  &  110011101100001 & 1011    \\
                    9  &  110011101100000 & 1000    \\
                    a  &  1000 &       0100000101   \\
                    A  &  10010001 &                \\
                    B  &  1011011010 &              \\
                    b  &  1100110 &                 \\
                    c  &  100110 &    0100010111    \\
                    C  &  1011010110 &              \\
                    D  &  0000110111 &                  \\
                    d  &  11000 &   010001010           \\
                    e  &  001 &    01000111             \\
                    E  &  100100001 &  010000010001     \\
                    F  &  000010110 & 0100010011010     \\
                    f  &  110010 &     0100001001   \\
                    g  &  100111 &     010000011    \\
                    G  &  11001110111 &             \\
                    h  &  0101 &     010000101      \\
                    H  &  101101100 &               \\
                    I  &  00001010 &                \\
                    i  &  0001 &    01000000        \\
                    J  &  00001011100 &             \\
                    j  &  10110110110 &             \\
                    K  &  101101101111 &            \\
                    k  &  11001111 &                \\
                    L  &  000011000 &               \\
                    l  &  01101 &   0100010000      \\
                    M  &  000011001 &               \\
                    m  &  101111 &   01000110110    \\
                    n  &  0100 &    010001100       \\
                    N  &  1100111000 &              \\
                    o  &  0111 &    010000110       \\
                    O  &  100100000 &               \\
                    p  &  011000 &  01000100101     \\
                    P  &  1011010111 &              \\
                    q  &  0000101111 &              \\
                    Q  &  1011010101011000 &        \\
                    R  &  1011010100 &              \\
                    r  &  11010 &   0100010001      \\
                    S  &  101101000 &  0100010011011\\
                    s  &  11011 &    0100011010     \\
                    T  &  0000100 &    010001011010 \\
                    t  &  1010 &   010000111        \\
                    u  &  00000 &    01000110111    \\
                    U  &  101101101110 &            \\
                    v  &  0000111 &    010001011011 \\
                    V  &  101101010100 &            \\
                    w  &  101100 &    01000101100   \\
                    W  &  101101001 &               \\
                    x  &  000011010 &               \\
                    X  &  11001110110001 &          \\
                    Y  &  00001101101 &             \\
                    y  &  011001 &   010001001100   \\
                    z  &  00001101100 &             \\
                    Z  &  1011010101011001 &        \\
                    \#  &              & 0100001000  \\
                    !  & 110011101101  &            \\
                    "  & 10110111 &                 \\
                    '  & 11001110101 &              \\ 
                    (  & 110011101100111 &          \\
                    )  & 110011101100110 &          \\
                    ,  & 100101 & 010000010000      \\
                    -  & 11001110100 &              \\
                    .  & 1001001 & 01000100111      \\
                    :  & 11001110011 & 01000100100  \\
                    ;  & 11001110010 & 01000001001  \\
                    ?  & 10110101011&               \\
                    ENTER & 101110 & 1101            \\
                    blackspace   & 111 & 111        \\
            \hline  % \hline 在此行下面画一横线
            \end{supertabular}
            \end{center}
        
\end{document}
